\documentclass[a4paper, 12pt]{article}
\usepackage[utf8]{inputenc}
\usepackage{graphicx}
\usepackage{hyperref}
\usepackage{listings} % Für Codebeispiele
\usepackage{amsmath} % Mathematische Formeln
\usepackage{caption}
\usepackage{subcaption}

%opening
\title{Informatic-project: Learning application for algorithm and data structures}
\author{Lorenzo Wendland, Dennis Kaufman, Dwipa Flügel}
\date{April 9, 2025}

\begin{document}

\maketitle

\begin{abstract}
This documentation describes the planning, implementation and results of the informatic project: "learn application for algorithm and data-structures".

\end{abstract}

\tableofcontents % automatisch generiertes Inhaltsverzeichnis

\section{Introduction}
\subsection{Motivation}
Why was this project picked? What's the goal of this application?

\subsection{Objective}
Which functionalities should be reached?

\subsection{Technologies}
Which program languages, frameworks or tools are being used?

\section{Project planning}
\subsection{Methodology}
Use of procedure (e.g. Scrum, Kanban)

\subsection{Architecture}
Description of software architecture with diagrams

\section{Implementation}
\subsection{Technical details}
Detailed description of the implementation

\subsection{Code examples}
\begin{lstlisting}[language=Python, caption=example code in Python]
	def example():
		print("Hello, World!")
\end{lstlisting}

\subsection{Problems and solutions}
Challenges that occured and how we solved them

\section{Results and reflection}
\subsection{Test results}
Which tests were carried out and which results were achieved

\subsection{Prospects}
Possible advancements and future improvements

\section{Appendix}
\subsection{Sources}
If external sources were used

\end{document}
